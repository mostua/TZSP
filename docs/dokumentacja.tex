\documentclass[a4paper,twoside,10pt]{article}

\usepackage[polish]{babel}
\usepackage[utf8]{inputenc}
\usepackage{polski}
\usepackage[T1]{fontenc}
\usepackage{fullpage}
\setlength{\parindent}{0pt}

\renewcommand{\labelenumii}{\theenumii}
\renewcommand{\theenumii}{\theenumi.\arabic{enumii}.}

\author{Jan \textsc{Wiśniewski}, Adam \textsc{Mościcki}, Edwin \textsc{Jarosiński}}
\title{Dokumentacja projektu z przedmiotu Podstawy Sztucznej Inteligencji}

\begin{document}
\maketitle
 
\section{Cel projektu}
	Celem projektu jest stworzenie aplikacji pozwalającej na generowanie "magicznych kwadratów" przy użyciu algorytmów ewolucyjnych.
\section{Założenia wstępne}
	Przyjmujemy następującą definicję magicznego kwadratu:
 tablica składająca się z $ n $ wierszy i $ n $ kolumn $ (n>2) $, w którą wpisano $ n^2 $ różnych nie powtarzających się dodatnich liczb naturalnych (od 1 do $n^2$) w ten sposób (użytkownik ma możliwość wyboru), że:
 \begin{itemize}
  \item suma liczb w każdym wierszu, w każdej kolumnie i w każdej przekątnej jest taka sama,
  \item suma liczb w każdym wierszu, w każdej kolumnie jest taka sama,
 \end{itemize}
 i wynosi $\frac{n^2\cdot(n^2+1)}{2*n}$
\section{Funkcja dopasowania}
Metoda prezentuje się następująco:
		\begin{itemize}
		\item wyliczenie oczekiwanej wartości sum w kolumnach, wierszach i przekątnych: $ s = (1 + n^2) \frac{n}{2} $
		\item policzenie sum w kolumnach, wierszach i przekątnych(jeżeli przekątne są ważne dla typu kwadratu)
		\item policzenie dla każdej z tych sum wartości absolutnej z różnicy tej sumy i wartości oczekiwanej
		\item zsumowanie wartości absolutnych
	\end{itemize}
\section{Algorytm ewolucyjny}
Zastosowany przez nas algorytm to $\mu$ +  $\lambda$. A jego pseudokod jest następujący:
		\begin{enumerate}
			\item Wylosuj populacje początkową P o rozumiarze $\mu$ osobników postaci $<x,\sigma>$, gdzie $x$ oznacza osobnika, a $\sigma$ prawdopodobienstwo jego mutacji (wylosowane z przedziału podanego przez użytkownika),
			\item Wybierz z P za pomocą odpowiedniej strategi $\lambda$ osobników, którzy będą się rozmnażać jako populacje tymczasową T.
			\item Zastosuj krzyżowanie dla populacji T, oraz mutację osobnika z prawdopodbieństwem $\sigma$  dla kwadratu (każdego z osobna) - wynik pouplacja R.
			\item $P := P \cup R$.
			\item Jeśli znaleziono magiczny kwadrat zakończ algorytm.
		\end{enumerate}
\section{Operator krzyżowania}
	Najważniejszą decyzją do podjęcia przy realizacji algorytmu ewolucyjnego jest wybór operatora krzyżowania. Jako że w podstawowej reprezentacji magiczny kwadrat jest macierzą, a wartość funkcji dopasowania zależy od sum kolumn wierszy i przekątnych, należy bardzo dokładnie przeanalizować dostępne opcje. Po rozważeniu różnych możliwości zdecydowaliśmy się na implementację dwóch różnych operatorów:
		\subsection{Zamiana przekątnych}
		\begin{enumerate}
			\item utworzenie dziecka - przepisanie pól rodzica A
			\item wybór jeszcze nie wybranego pola rodzica B znajdującego się na jednej z jego przekątnych
			\item przepisanie wybranego pola z rodzica B do odpowiadającego mu pola dziecka
			\item jeżeli w dziecku występuje druga taka sama liczba jak ta właśnie pobrana od rodzica B to jest ona podmieniana na liczbę która została przez nią nadpisana
			\item jeżeli rodzic B posiada jakieś nie wybrane pola na przekątnych to skok do punktu nr 2
			\item koniec
		\end{enumerate}
		\subsection{Zamiana wierszy albo kolumn}
		Opisany algorytm operuje na kolumnach. Zamiana wierszy jest dokonywana analogicznie.
		\begin{enumerate}
			\item utworzenie dziecka - przepisanie pól rodzica A
			\item losowy wybór $ \lfloor\frac{N}{2}\rfloor $ kolumn
			\item wybór jeszcze nie wybranego pola rodzica B znajdującego się w wybranych kolumnach
			\item przepisanie wybranego pola z rodzica B do odpowiadającego mu pola dziecka
			\item jeżeli w dziecku występuje druga taka sama liczba jak ta właśnie pobrana od rodzica B to jest ona podmieniana na liczbę która została przez nią nadpisana
			\item jeżeli rodzic B posiada jakieś nie wybrane pola na wybranych kolumnach to skok do punktu nr 3
			\item koniec
		\end{enumerate}
		Niezależnie od metody krzyżowanie rodziców wartość $ \sigma $ dla dziecka jest wyliczana jako średnia arytmetyczna z wartości $ \sigma$ obojga rodziców.
\section{Operator mutacji}
Dostępne są 2 operatory testowania:
\begin{enumerate}
\item zamiana dwóch punktów,
\item zamiana dwóch kolumn.
\end{enumerate}
Warto zauważyć że 2 z operatorów nie zamienia sumy w kolumnach i wierszach (a tylko na przekątnych), zmiana wartości funkcji przystowania w tym przypadku jest więc zależna tylko od operatora krzyżowania.
\section{Wybór osobników do krzyżownia}
Selekcja nastepuje w trzech możliwych sposobach:
\begin{itemize}
\item Rankingowa:
Osobniki są wybierane z prawdopodobieństwm proporcjonalnym do rankingu ich funkcji przystowania.
\item Ruletki:
Ta metoda jest zmodyfikowaną wersją standardowej wersji, polega na wyborze osobników z pradopodbieństwem $1 - \frac{przystowanie(osobnik) - przystosowanie(najlepszy)}{przystosowanie(osobik)}$.
\item Najlpeszych:
Brane jest $\mu$ najlepszych osobików (jeżeli $\mu$ jest wieksze od wielkości populacji, to branie jest rozpoczynane od końca).
\end{itemize}
Praktyka pokazuje, że ostatni z tych typów daje najlepsze rezultaty, ponieważ często jest tak, że jedna kluczowa zamiana potrafi osobnika bardzo złego zamienić na bardzo dobrego.
\section{Optymalizacje}
Użytkownik ma do wyboru 2 optymalizacje. Pierwsza polega na dodaniu nowych osobników co którąś iterację (czasem pozwala to wyjść z lokalnego maksimum). Drugie jest przyspieszeniem symulacji poprzez ogranicznenie populacji do jakichś rozmiarów (dzięki czemu oszczędzamy czas nie przetwarząjac słabo przystosowanych osobników).
\section{Testowanie}
Testowaliśmy naszą symulację z różnymi algorytmami i wartościami stałych. Najlepsze wyniki uzyskiwaliśmy przy wykorzystaniu metody Najlepszych jako użytej metody selekcji. Skuteczność działania naszego algorytmu często poprawia dodanie nowych osobników do populacji co jakiś czas.\\
Sto prób dla kwadratu z liczącymi się przekątnymi o rozmiarze 3, z parametrach $\mu$ równemu 500 oraz $\lambda$ równemu 1000, $\sigma$ stałą równa $80\%$ dla selekcji najlepszych przy mutacji zamieniającej 2 pola zakończyło się sukcesem w 100 przypadkach (3 z nich wychodziły poza 10 iteracje). \\
Jeden z otrzymanych kwadratów: \\
4 8 3 \\
2 6 7 \\
9 1 5 \\
20 prób dla kwadratu o rozmiarze 4 przy tych samych parametrach dało sukces w 5 przypadkach. \\
Jeden z otrzymanych kwadratów: \\
7 6 12 9 \\
14 15 1 4\\
2 3 16 13\\
11 10 5 8 
\section{Wnioski}
Dla tablic o wielkości nie przekraczającej 5 da się bez większych problemów generować magiczne kwadraty (czasem po kilkunastu próbach). Z powodu problemów ze znalezieniem optymalnego operatora krzyżowania stosowanie algorytmów ewolucyjnych do tworzenia magicznych jest mało optymalne i nie powinno stanowić większego problemu znalezienie strategii heurystycznej radzącej sobie z tym problemem zdecydowanie lepiej.
\end{document}

